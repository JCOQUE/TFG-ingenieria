\pagestyle{plain}

\section{Introducción}
Cada año, España cuenta con más personas a las que le gustaría abrir su propio negocio. Según el Directorio Central de Empresas (DIRCE), en España, hay 3.2 millones de empresas activas \parencite{ine}. Si también se tienen en cuenta las personas españolas a las que le gustaría abrir su propio negocio, estas son tres de cada diez \parencite{CEAJE}.

El objetivo de toda persona con un negocio ya activo, o con la idea de emprenderlo, es que este negocio sea próspero, i.e. que genere beneficios. Para ello, mantener un buen funcionamiento de las cuentas y de la trayectoria financiera de la empresa resulta esencial \parencite{importanciaContabilidad}. Estas tareas ---entre otras--- se encarga de llevarlas a cabo el departamento de \B{contabilidad}  de una empresa. 

Esta actividad \parencite{historiaContabilidad} lleva acompañando al ser humano mucho antes de lo que uno se puede imaginar. Investigaciones de los años noventa, permitieron a diversos arqueólogos e historiadores afirmar que los primeros documentos escritos descubiertos\fnm\ por el ser humano no eran otra cosa que tablas de arcilla con números y cuentas. Este descubrimiento permitió a estos investigadores llegar a la conclusión de que, el nacimiento de la escritura en el año $3.000$ a.c. no fue por el deseo de transmitir conocimiento a generaciones futuras ---como se pensaba en primera instancia---, sino de la mera necesidad de conservar cuentas de procesos productivos y administrativos.
\fnt{Hace más de $5.000$ años.}

La contabilidad de hoy en día se rige por el método de la partida doble\fnm. Curiosamente, fue España, con las pragmáticas de Cigales de $1549$ y de Madrid de $1552$, el país que introdujo este método al obligar a los comerciantes de aquella época a llevar libros de cuentas consigo mismos mediante el método de partida doble \parencite{icac}.
\fnt{Principio fundamental en contabilidad que establece que cada transacción económica afecta al menos a dos cuentas (deudora y acreedora), y que la suma de los débitos siempre debe ser igual a 0.}

No fue hasta finales del siglo pasado que, con el desarrollo de la tecnología software, empresas empezaron a ver el potencial de trasladar sus registros contables de libros a una herramienta informática diseñada para hacer esta tarea de una manera más amena y eficiente. La primera de estas herramientas fue \ti{Peachtree Accounting}, desarrollada por la empresa Peachtree Software en 1975.

Actualmente, la actividad de contabilidad, salvo en casos muy específicos de algunas pequeñas empresas, no se lleva a papel. La mayoría de empresas, en cambio, se apoyan del software para tareas contables. Hoy en día no existe una, sino múltiples herramientas informáticas de las que el usuario dispone. Un ejemplo de una de ellas es Odoo, pero hay muchas otras.

\subsection{Justificación del contexto}
La contabilidad explica la historia de una empresa con números. Esto lo consigue llevando un orden claro y preciso de las cuentas, actividades, recursos y dinero. En otras palabras, la contabilidad se encarga de llevar un registro de todas las operaciones de una empresa; de todo el dinero que entra y sale de ella. Esto, no solo es necesario para respaldar los datos económicos de la empresa ante las Agencias Tributarias, sino que también provee información del estado financiero actual de la empresa; así como su evolución durante un periodo determinado de tiempo. Por ende, este departamento es de suma importancia en cualquier empresa, sin importar el tamaño de esta.

El éxito de un negocio ---ya sea conseguirlo o aumentarlo--- es determinado por las decisiones que se tomen en él. La mayoría de estas elecciones tienen como objetivo maximizar el beneficio de la empresa. Es por ello que la contabilidad es vital para informar de, o bien del resultado de decisiones tomadas en el \B{pasado}, o de decisiones por tomar en el \B{presente} que tendrán un impacto en el \B{futuro}.

Para determinar si la empresa se está beneficiando de una decisión tomada en el pasado, la contabilidad aporta datos sobre la situación financiera pasada y actual de esta. Al estudio de estos datos pasados y presentes se le conoce como \ti{Business Intelligence} (BI). De esta manera, una corporación puede saber cuanto antes las consecuencias que han conllevado sus decisiones; y, por ende, si por un lado debe seguir por el mismo camino, u optar por otro para tratar de revertir la situación.

Respecto a la toma de decisiones presentes con repercusión en el futuro, la contabilidad también juega un papel crucial. Conocer el estado financiero actual de la empresa es fundamental para saber el límite máximo de riesgo e inversión que una empresa puede asumir e incurrir respectivamente. Aunque riesgo e incertidumbre siempre va a haber a la hora de tomar una decisión con impactos futuros, un correcto análisis de los datos que una empresa posee puede reducir este riesgo a que sea el mínimo posible. El análisis del impacto de estos datos en el futuro ---i.e. predicción--- se le conoce como \B{BA} (\ti{Business Analytics}).

A estas tomas de decisión basadas en datos se le  conoce como \ti{data-driven decision making}.


\subsection{Planteamiento del problema}
Como ya se ha mencionado con anterioridad, la contabilidad es la encargada de llevar, de la manera más rigurosa posible, todas entradas y salidas de dinero de una empresa. La actividad diaria de una empresa es la de realizar operaciones y transacciones internas con clientes, proveedores, etc. Es decir, no hace falta que una empresa sea muy grande para que esta genere un gran número de flujos de entrada y salida todos los días.  

Estos flujos de entrada y salida son datos; un gran número de datos que una empresa obtiene día a día. Estos datos, a menos que se procesen y analicen, no sirven de mucho y es muy difícil extraer conclusiones a partir de ellos que puedan resultar útiles para la compañía. Es pues, a través de un análisis de estos datos, cuando se obtiene información relevante acerca de ellos: proveedores a los que más se compra, deudas de la empresa, sectores en los que la empresa tiene más éxito, clientes que más compran, etc.

El análisis de este gran volumen de datos, que se producen en un breve periodo de tiempo, se vuelve un reto para las empresas. Además, a medida que estas van creciendo, más flujos de entrada y salida generan a diario; lo cual supone más datos y esto implica mayor dificultad a la hora de estudiarlos. 

Para obtener hallazgos significativos a partir de ellos ---y por ende, permitir a la empresa prosperar---, los dos análisis comentados en el apartado anterior ---el análisis de la situación pasada y actual, así como una proyección hacia el futuro--- son cruciales.

Por una parte, BI ayuda a explicar la situación actual de la empresa y cómo ha llegado a estar donde se encuentra actualmente. Es decir, la evolución de esta en un determinado periodo de tiempo. Algunas de las preguntas que podría responder BI son: ¿por qué este producto (no) ha tenido éxito?  ¿Por qué han aumentado las ventas en este sector y disminuido en este otro? ¿Fue correcta una decisión en concreto?

Por otra parte, BA permite analizar tendencias y averiguar hacia dónde debe la empresa centrar sus esfuerzos y atención. Las preguntas que podría responder son: ¿qué evolución debería experimentar la empresa en un periodo de tiempo determinado? ¿Se debería contratar más personal porque se espera que la empresa evolucione considerablemente en el futuro? ¿Cómo se espera que evolucionen las tendencias de distintos sectores?

Todas estas preguntas no son triviales de responder y, para poder contestarlas de la manera más rigurosa y fiable posible, se requieren de herramientas software capaces de procesar una gran cantidad de datos (i.e. \ti{Big Data}).


\subsection{Objetivos del trabajo}

Este trabajo tiene como objetivo llevar a cabo una herramienta BI/BA para el análisis de datos contables. Para ello, se  espera realizar un \ti{dashboard} en donde mostrar los datos. El trabajo parte del punto de vista que quien usa la herramienta es un directivo ---o cualquier otro puesto con responsabilidad decisiva--- al que le interesa que se le muestren datos contables de una empresa de una manera fácil e intuitiva. De esta manera, las tomas de decisión se vuelven más claras y más sencillas. Para ello, esta herramienta deberá contar con:
\begin{itemize}
    \item Interactividad para que el usuario pueda ver distintos datos, y cómo afectan unos a otros.
    \item Una automatización para que el proceso de extracción de datos sea lo más ameno posible.
    \item Arquitectura enfocada a \ti{Big Data} preparada para poder procesar una gran cantidad de  datos.
    \item Un alojamiento seguro de los datos, solo visible para aquellas personas con permisos.
    \item Esta herramienta deberá ser accesible desde internet.
    \item Además, se mostrarán estimaciones futuras. Para ello, sería ideal un seguimiento de los modelos de aprendizaje automático, así como un registro de estos, sus métricas, sus parámetros y otra información relevante.
    \item Por último, y no por ello menos importante, la herramienta deberá ser sencilla de utilizar e intuitiva.
\end{itemize}