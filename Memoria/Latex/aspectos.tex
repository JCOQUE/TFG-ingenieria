\pagestyle{plain}

\section{Aspectos metodológicos}

\subsection{Metodología}

\subsection{Tecnologías aplicadas}
\begin{nota}
Aquí, tengo que poner por qué he elegido unos modelos y no otros. Decir que pese a que hay transformers mas actuales pasa time series (fedformer y autoformer, cuyuos paper tengo descargados) se utiliza informer porque es open source y tienen un repositorio de github donde viene cómo hacer fine tunning de una manera sencilla (explicar en apéndie qué es fine-tuning). Y bueno defender al informer, mostrando trablas indicando que muesta mejores resultados que LSTM y otras variantes de transformer...
\end{nota}
\begin{comment}
\begin{itemize}
    \item BI (nada específico. Distintas herramientas en pocas líneas)
        \begin{itemize}
            \item{Gráficos que se vayan a emplear (de barras, de puntos, etc.)}
        \end{itemize}
    \item BA
        \begin{itemize}
            \item Odoo
        \end{itemize}
\end{itemize}
\end{comment}